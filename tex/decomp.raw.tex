\documentclass{article}

\usepackage{parskip,enumerate,amsthm,amsmath}

\newtheorem*{theorem}{Theorem}
\newtheorem*{proposition}{Proposition}

\title{Computing Decompositions of Linear Representations Into Irreducibles}
\author{Kaashif Hymabaccus}

\begin{document}
\maketitle
\tableofcontents
\newpage

\section{Introduction}

Given a group $G$, a linear representation of this group is a
homomorphism $\rho$ from $G$ to $GL(V)$, the group of linear
automorphisms of a vector space $V$.

I will limit myself to the case where $G$ is finite and $V$ is finite
dimensional. This can be done since given any such representation of a
finite $G$, we can just take the subgroup generated by all images of
elements of $G$.

\section{Computing the canonical decomposition}

Henceforth, given a representation $\rho : G \to GL(V)$, we will also
refer to $V$ as a representation of $G$.

Thus a decomposition of $V$ into irreducibles is:

$$V = U_1 \oplus \ldots \oplus U_m$$

where each $U_k$ is irreducible.

The canonical decomposition of $V$ is obtained by decomposing $V$ into
irreducibles as above, then collecting together the isomorphic
irreducible representations. Let $W_1$ to $W_n$ be the list of all
irreducible representations of $G$, then the canonical decomposition is:

$$V = V_1 \oplus \ldots \oplus V_n$$

Where $V_i$ is the direct sum of a number (possibly zero) of
irreducible representations isomorphic to $W_i$.

Serre's theorem is as follows:
\vspace{0.5em}

\begin{theorem}[Theorem 8 in Serre]\ \\
\vspace{-1em}
\begin{enumerate}[(i)]
  \item The decomposition $V = V_1 \oplus \ldots \oplus V_n$ does not
    depend on the initially chosen decomposition of $V$ into
    irreducible elements.
  \item The projection $p_i$ of $V$ into $V_i$ associated with this
    decomposition is given by the formula:

    $$p_i = \frac{n_i}{g} \sum_{t \in G}\chi_i(t)^*\rho_t$$

    Where $g = |G|$, $n_i$ is the degree of $W_i$ and $\chi_i$ is its
    character.
\end{enumerate}
\end{theorem}
\begin{proof}
See Serre Theorem 8.
\end{proof}

This can be implemented as a function in GAP transforming the list of
irreducible representations of $G$ into the list of projections of $V$
into each component of the canonical decomposition, then finally into
the list of the components (by taking images).

Notice that, although we use various facts about the full irreducible
decomposition, we do not need to know it to work out the canonical
decomposition.

\section{Computing the irreducible decomposition}

Now that we have the list of $V_i$, the next step is breaking this
down into its components:

$$V_i \cong W_i \oplus \ldots \oplus W_i$$

Of course we know what it is up to isomorphism, but this is not (very)
useful. Serre gives us an easy way to concretely decompose $V_i$.

Each component arises as the base space of the codomain of a
homomorphism $\rho : G \to GL(W_i)$. If we pick a basis $(e_1, \ldots,
e_n)$ for $W_i$, we can write this representation in what is called
matrix form. For each element $s \in G$, if $\rho_s$ is it's image,
then write $R_s$ for the matrix of $\rho_s$ in the basis $(e_j)$.

To keep with Serre's notation, write $r_{\alpha\beta}(s)$ for
$(R_s)_{\alpha\beta}$, where $\alpha$ and $\beta$ index over the rows
and columns of the matrix. Since we are dealing with finite groups, we
can always ensure this matrix is also finite.

Define:

\begin{equation}
p_{\alpha\beta} = \frac{n}{g}\sum_{t \in G} r_{\beta\alpha}(t^{-1})\rho_t \tag{$*$}
\end{equation}

\begin{proposition}[Proposition 8 in Serre]\ \\
\vspace{-1em}
\begin{enumerate}[(a)]
  \item The map $p_{\alpha\alpha}$ is a projection; it is zero on the
    $V_j$ for $j \neq i$. Its image $V_{i,\alpha}$ is contained in
    $V_i$ and $V_i$ is the direct sum of the $V_{i,\alpha}$ for $1
    \leq \alpha \leq n$. We have $p_i = \sum_\alpha p_{\alpha\alpha}$.

  \item The linear map $p_{\alpha\beta}$ is zero on the $V_j$, $j \neq
    i$ as well as on the $V_{i,\gamma}$ for $\gamma \neq \beta$; it
    defines an isomorphism from $V_{i, \beta}$ onto $V_{i, \alpha}$.

  \item Let $x_1 \neq 0$ be an element of $V_{i,1}$ and let $x_\alpha
    = p_{\alpha,1} \in V_{i,\alpha}$. The $x_\alpha$ are linearly
    independent and generate a vector subspace $W(x_1)$ stable under
    $G$ and of dimension $n$. For each $s \in G$, we have:

    $$\rho_s(x_\alpha) = \sum_\beta r_{\beta\alpha}(s)x_\beta$$

    (in particular, $W(x_1)$ is isomorphic to $W_i$)

  \item If $(x_1^{(1)}, \ldots, x_1^{(m)})$ is a basis of $V_{i,1}$,
    the representation $V_i$ is the direct sum of the
    subrepresentations $W(x_1^{(1)}), \ldots, W(x_1^{(m)})$ defined in
    (c).
\end{enumerate}

(Thus the choice of a basis of $V_{i,1}$ gives a decomposition of
$V_i$ into a direct sum of representations isomorphic to $W_i$).
\end{proposition}

\begin{proof}
See Serre Proposition 8.
\end{proof}

Having previously computed the canonical decompositions, we can use
Proposition 8 to decompose each component.

\section{Benchmarks?}

\section{Optimisation}

Notice that, for large groups, this algorithm will be very slow. In
calculating projections, there is a lot of summation over elements of
$G$. This is wasted effort, since we know that any character is a
class function, so we need only calculate the summand once for each
class.

This reduces the time taken greatly for some groups with large numbers
of elements but small numbers of conjugacy classes, for example the
dihedral groups with $2n$ elements but only $(n+k)/2$ classes (where
$k$ is either 3 or 6).

\section{More benchmarks?}

\end{document}
